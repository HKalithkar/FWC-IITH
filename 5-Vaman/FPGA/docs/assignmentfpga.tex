\documentclass{article}
\usepackage{multirow}
\usepackage{blindtext}
\usepackage{amsmath}
\usepackage{capt-of}
\usepackage{circuitikz}
\usepackage{listings}
\usepackage{./karnaugh-map}
\usetikzlibrary{shapes.geometric}

\lstset{
	language=C++,
	basicstyle=\ttfamily\footnotesize,
	breaklines=true,
	frame=lines
}

\title{Implementation of given Boolean Logics using Vaman FPGA}
\date{April 2023}
\author{Sai Harshith Kalithkar\\harshith.work@gmail.com\\FWC22118\\IIT Hyderabad-Future Wireless Communication Assignment-4.2}

\begin{document}
\maketitle
	\tableofcontents
\pagebreak
\section{Problem}
	(GATE EC-2022)\\
	Q.6. Which one of the following is NOT a valid identity?
\\

\begin{enumerate}
	\item \begin{equation} (x \oplus y) \oplus z = x \oplus (y \oplus z) \end{equation}
	\item \begin{equation} (x + y) \oplus z = x \oplus (y + z) \end{equation}
	\item \begin{equation} x \oplus y = x + y, if xy = 0  \end{equation}
		\item \begin{equation} x \oplus y = \sim(xy + (\sim(x) \sim(y)) \end{equation} \\
\end{enumerate}

\section{Introduction}
		The Aim is to implement the above four logics and find out the one which is not true or not valid.\\
\section{Components}
	\begin{enumerate}
		\item Vaman Board
		\item LEDs - 4
		\item Breadboard
		\item Jumper Wires (M-F) and (M-M) \\
	\end{enumerate}

\section{Hardware}
	A total of 4 LEDs will be used in hardware each one representing one logic. The logics will keep running in loop. The X, Y and Z values will be given as inputs to the Vaman Board. Whenever the logic is true the led will glow. And whenever the logic is false the led turns off. \\
	One out of the given 4 logic is false for some cmbination of X, Y, Z values. While the remaining 3 logics is true for all the combination of X, Y, Z values. This way we can find the odd one out. \\

	The truth table for the circuit is given in below table \\
	\begin{table}[h]
		\begin{center}
	\begin{tabular}{|c|c|c|c|c|c|c|}
        \hline X & Y & Z & L1 & L2 & L3 & L3 \\
        \hline 0 & 0 & 0 & 1 & 1 & 1 & 1 \\
        \hline 0 & 0 & 1 & 1 & 1 & 1 & 1 \\
        \hline 0 & 1 & 0 & 1 & 1 & 1 & 1 \\
        \hline 0 & 1 & 1 & 1 & 0 & 1 & 1 \\
	\hline 1 & 0 & 0 & 1 & 1 & 1 & 1 \\
	\hline 1 & 0 & 1 & 1 & 1 & 1 & 1 \\
	\hline 1 & 1 & 0 & 1 & 0 & - & 1 \\
	\hline 1 & 1 & 1 & 1 & 1 & - & 1 \\
	\hline
        \end{tabular}

		\caption{Truth Table}
		\label{table:2}
		\end{center}
	\end{table}

\section{Software}
	The Arduino code for the given circuit using IC 7474 is \\
	\lstinputlisting{helloworldfpga.v}
	The pin connections for the above code is
	\lstinputlisting{quickfeather.pcf}
\end{document}
